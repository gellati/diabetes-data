\documentclass[letterpaper, 12pt]{article}
\usepackage[dvips]{graphicx}
\usepackage[T1]{fontenc}
\usepackage{times}
\usepackage{amsmath,amsfonts,amssymb}
\usepackage{inputenc}
\usepackage{euscript}
\usepackage{pstricks}
\usepackage{pst-node}
\usepackage{epsf}
\usepackage{fancyhdr}

\setlength{\parindent}{0mm}
\addtolength{\textwidth}{3mm}
\addtolength{\textheight}{12mm}
\addtolength{\voffset}{-6mm}

\newcommand{\EqRef}[1]{Equation \ref{#1}}
\newcommand{\FigRef}[1]{Figure \ref{#1}}
\newcommand{\TabRef}[1]{Table \ref{#1}}
\newcommand{\Symbf}[1]{\mbox{\boldmath{$#1$}}}
\newcommand{\Section}[1]{\section{#1} \vspace{-3.5mm}}
\newcommand{\SubSection}[1]{\subsection*{#1} \vspace{-3mm}}
\newcommand{\SubSubSection}[1]{\subsubsection*{#1} \vspace{-3mm}}

% Custom page style.
\pagestyle{fancy}
\fancyhf{}
\fancyhead[RO]{
  \begin{pspicture}(0, 0)(0, 0)
    \rput[L]( 0.0,  6pt){\thepage}
  \end{pspicture}
}
\fancyhead[LO]{\rightmark}
\renewcommand{\sectionmark}[1]{\markright{}}
\renewcommand{\headrulewidth}{0pt}
\renewcommand{\footrulewidth}{0pt}
\fancypagestyle{plain} {\fancyhead{}}

\title{Himmeli v3.0 user's guide}
        \author{Ville-Petteri M�kinen \\
        \small Laboratory of Computational Engineering \\
        \small Helsinki University of Technology}



\begin{document}
\maketitle
\setlength{\parskip}{-2mm}
\tableofcontents
\setlength{\parskip}{3mm}

\Section{Getting started}
Automatic graph drawing is perhaps the most widely applicable visualization
aid that the advent of computers has brought to the scientific community. For
a researcher, a clear picture of a complicated set of relationships helps to
identify interesting features or to understand the large-scale nature of the
problem at hand. The program Himmeli was developed to meet this need or, more
specifically, to produce two-dimensional representations of weighted,
directed and non-planar graphs. It is based on a force-directed algorithm
that is capable of making acceptable vertex layouts of almost every type of
network.

Himmeli has no graphical user interface\footnote{The installation of 
Himmeli is similar to that of CraneFoot software, see CraneFoot user's
guide for details.} but creates a PostScript file of the
graph instead. You can either print the file directly or open it with a
viewer software such as GhostView. In addition, it is possible to convert
the EPS-file\footnote{encapsulated PostScript} to the portable document
format (PDF); in most Linux platforms the command \texttt{ps2pdf} works well
for the fixed paper sizes. Another useful tool is \texttt{eps2pdf} by Wouter
Kager and many image processing software (e.g. Gimp) accept EPS, too. If you
are planning on using Himmeli for online graph drawing, the ImageMagick
graphics library can be used for conversions from EPS to other image formats.

Himmeli requires two input files to work. The first, referred to as the
\textit{configuration file}, contains the instructions that specify
where to look for the graph data and how to visualize it. Furthermore, the
name of the configuration file is the only command line argument that Himmeli
accepts, everything else is defined by the instructions. \textbf{If you call
Himmeli without any command line arguments, a concise listing of all features
is printed on the screen.}

\begin{figure}[ht]
\begin{pspicture}(0, 0)(\textwidth, 9)
  \rput[lb](2.0, -1.0){\scalebox{0.8}{\epsfbox{figure1.eps}}}
  %\psframe(0, 0)(\textwidth, 9)
\end{pspicture}
\caption{An undecorated drawing of the largest component of a non-planar
  random graph.}
\label{fig1}
\end{figure}

The second file, referred to as the \textit{edge file}, contains the edges of
the graph in tabulated ASCII text format. The first non-empty line of the file
must contain the headings of the variable columns and the subsequent lines
specify the graph topology. For each edge, you have to list at least the
names of the endpoints on a single line in order for the edge to be registered
by Himmeli. For instance, consider the first few rows and columns of the
listing for the graph in Figure 1, written as

\textit{edges.txt}: \\
\small
\verb|HEAD    TAIL    WEIGHT  ..| \\
\verb|v0      v1      0.100| \\
\verb|v0      v8      0.100| \\
\verb|v0      v9      0.100| \\
\verb|v0      v14     0.100| \\
\verb|v0      v18     0.100| \\
\verb|v0      v19     0.100| \\
\verb|v1      v4      0.200| \\
\verb|v1      v10     0.200| \\
\verb|v1      v12     0.200| \\
\verb|v2      v6      0.300| \\
\verb|v2      v9      0.300| \\
\verb|:|
\normalsize

Note that the columns are separated by horizontal tabulator characters that
are not explicitly displayed here. If you wish to use another character, use
the \texttt{Delimiter} instruction in your configuration file, details are at
the end of this guide. The above alone is not enough, however, since without
additional knowledge you cannot tell which column contains which variable.
Hence the configuration file is needed to convey the necessary information to
Himmeli. To produce \FigRef{fig1}, we need five instructions, formulated as

\textit{config.txt}: \\
\small
\verb|GraphName                myGraph| \\
\verb|EdgeFile                 edges.txt| \\
\verb|EdgeHeadVariable         HEAD| \\
\verb|EdgeTailVariable         TAIL| \\
\verb|EdgeWeightVariable       WEIGHT .| \\
\normalsize

The first instruction \texttt{GraphName} sets the common part of output
file names. Himmeli does not assume that the graph is connected, and it will
print each component on a separate page (or file if so instructed, see
\texttt{FigureLimit} for details). Largest components are printed first as
can be witnessed in the output file \textit{myGraph.ps}. 
The next line tells the name of the edge file, and the remaining two
instructions indicate the way the relevant columns should be used by Himmeli.
If you do not have the weights, simply omit the last instruction and Himmeli
will assign a uniform weight to each edge. Note that it is not necessary to
use the same column headings as in the example, you could use 'Source' and
'Sink' instead of 'TAIL' and 'HEAD' or any other. In addition, you do not
have to put the columns in any specific order. Finally, to ease the burden of
data setup, Himmeli detects the names of the vertices automatically from
the edge file so you do not have to provide any other information about the
graph to produce a basic layout. 

Figure \ref{fig1} could be improved, however, since at the moment the weights
and directions of edges are not displayed. Also, it would be interesting to
see the vertex strengths, that is, the sum of adjacent edge weights. You can
tell Himmeli to decorate the drawing automatically by setting
\texttt{DecorationMode} on in the configuration file. The colors, edge
widths and vertex sizes are now determined by the weights and strengths
(\FigRef{fig2}). The visual cues are normalized for each graph so they are
not suitable for comparing networks that have different strength
distributions. To make matters easier, you can influence the use of weights
by transformations and filtering, more on that later.

\begin{figure}[ht]
\begin{pspicture}(0, 0)(\textwidth, 9)
  \rput[lb](2.0, -1.0){\scalebox{0.8}{\epsfbox{figure2.eps}}}
  \psframe(0, 0)(\textwidth, 9)
\end{pspicture}
\caption{A decorated version of the weighted component in \FigRef{fig1}.}
\label{fig2}
\end{figure}

When looking \FigRef{fig2} closely, you will notice that the layout is not
identical with \FigRef{fig1}. Graphs are high-dimensional objects so putting
them on a 2D canvas is a brutal act. Furthermore, Himmeli's layout algorithm
has a stochastic component that finds a minimum stress configuration -- but
not necessarily the global minimum. Also, in this case the topology was
slightly changed by automatic weight normalization.
 
\Section{Customizing outlook}
Although Himmeli can automatically color the graph, it is often necessary
to display specific information that requires custom visualization. For this
reason, you can set the color, width and label for each individual edge in the
edge file. Suppose you are interested in vertices $v_{14}$ and $v_{16}$ and
wish to display only those edges that are adjacent to them. This can be
accomplished by adding an extra column\footnote{Himmeli also permits the use
of auxiliary edge files, a topic that is discussed in a separate section.} in
the edge file, written as

\textit{edges.txt}: \\
\small
\verb|HEAD    TAIL    WEIGHT  WIDTH ..| \\
\verb|:| \\
\verb|v11     v17     1.2     -1| \\
\verb|v11     v19     1.2     -1| \\
\verb|v12     v14     1.3| \\
\verb|v13     v18     1.4     -1| \\
\verb|v13     v14     1.4| \\
\verb|v16     v17     1.7| \\
\verb|:|
\normalsize

Rather than explicitly setting the widths of the target edges, you can assign
a negative width for the unwanted links instead. By doing this, Himmeli will
still automatically decorate the unaffected edges, but skip the drawing of
unwanted edges. It is important at this point to emphasize that the widths
or any other visualization instructions have no effect on the vertex
positioning algorithm, only the way the various objects are drawn will be
altered.

To distinguish the two vertices from the rest, you may wish to use a different
node symbol or color. For that, it is necessary to use a third input file,
referred to as the \textit{vertex file}. It has the same basic format as the
edge file, that is, a header line of variable names followed by vertex
records, each on a separate line. You need two columns: one for the vertex
name and the other for the symbol code, formulated as

\textit{vertices.txt}: \\
\small
\verb|NAME    GROUP| \\
\verb|v14     4| \\
\verb|v16     4| \\
\verb|:|
\normalsize

The number 4 stands for the diamond shape, a complete list of shapes is
depicted in \FigRef{fig5}. As before, you must use the configuration file to
pass this extra information to Himmeli, listed as

\textit{config.txt}: \\
\small
\verb|EdgeWidthVariable        WIDTH| \\
\verb|VertexFile               vertices.txt| \\
\verb|VertexNameVariable       NAME| \\
\verb|VertexShapeVariable      GROUP .|
\normalsize

The parameter \texttt{VertexNameVariable} indicates the column that contains
the names of the target vertices. Notice that the names in the edge file are
not all listed in the vertex file. In fact, Himmeli will detect only
those vertices already present in the edge file but, as seen here, you do not
have to include them all. Another important point is that you do not
have to fix the order of edges or vertices in the data files, Himmeli can
sort them automatically. The result is depicted in \FigRef{fig3}.

\begin{figure}[hb]
\begin{pspicture}(0, 0)(\textwidth, 5.0)
  \rput(7.0, 2.3){\scalebox{0.9}{\epsfbox{figure5.eps}}}
\end{pspicture}
\caption{Node shapes, patterns and some color definitions.}
\label{fig5}
\end{figure}

\begin{figure}[ht]
\begin{pspicture}(0, 0)(\textwidth, 8.5)
  \rput[lb](2.0, -1.0){\scalebox{0.8}{\epsfbox{figure3.eps}}}
  %\psframe(0, 0)(\textwidth, 8.5)
\end{pspicture}
\caption{A customized version of the component in \FigRef{fig1}.}
\label{fig3}
\end{figure}

\Section{Color and legend}
Suppose that next you want to display the incoming and outgoing edges with
different colors for the two vertices $v_{14}$ and $v_{16}$. Again, you start
by adding a column to the edge file and a corresponding instruction to the
configuration file, written as

\textit{edges.txt}: \\
\small
\verb|HEAD    TAIL    WEIGHT  WIDTH   COLOR| \\
\verb|v11     v17     1.2     -1| \\
\verb|v11     v19     1.2     -1| \\
\verb|v12     v14     1.3             000090| \\
\verb|v13     v18     1.4     -1| \\
\verb|v13     v14     1.4             000090| \\
\verb|v16     v17     1.7             900000| \\
\normalsize
\verb|:|

and you can further emphasize the direction with an arrow head by enabling
the \texttt{ArrowMode}. The human eye has three types of color receptors
and for this reason colors can be defined as a combination of red,
green and blue components. The values of the variable \texttt{COLOR} consist
of three two digit blocks, each with the range from 00-99 indicating the
strength of the primary color component. Here the aim was for a blue
(000090) and red (900000). This is similar to the way colors are defined in
HTML, for example, except that the decimal number system rather than the
hexadecimal is used for clarity.

When adding shapes and colors to the graph presentation it is important to
have a legend that describes the meanings of the various symbols. Himmeli
collects the visualization instructions from the configuration file and prints
the results on the edge of each page, provided that \texttt{LegendMode} is on.
Here it is instructive to add items that link the colors to edge directions
and the diamond shape to the target vertices. To do this (and the previous
enhancements), add the lines

\textit{config.txt}: \\
\small
\verb|ArrowMode           on| \\
\verb|LegendMode          on| \\
\verb|EdgeColorVariable   COLOR| \\
\verb|EdgeColorInfo       incoming   900000| \\
\verb|EdgeColorInfo       outgoing   000090| \\
\verb|VertexShapeInfo     target     4 |.
\normalsize

to the configuration file. The first value field of the ``info'' instructions
sets the label and the second specifies the corresponding shape or color, as
seen in \FigRef{fig4}. The info parameters do not affect the graph drawing
itself so you can add one even if you do not use that particular item in the
current presentation.

\begin{figure}[ht]
\begin{pspicture}(0, 0)(\textwidth, 9.5)
  \rput[lb](1.0, -0.5){\epsfbox{figure4.eps}}
  %\psframe(0, 0)(\textwidth, 9.5)
\end{pspicture}
\caption{A customized version of the component in \FigRef{fig1} and the
  respective legend.}
\label{fig4}
\end{figure}

\Section{Layout reconstruction}
Previously, the discussion was focused on the file formats and visual outlook
of the drawings but an equally important issue is the quality of the vertex
positioning. Himmeli employs a force-directed algorithm that assigns repulsive
and attractive forces between vertices based on their mutual distances and
connections. By simulating this artificial system of strings, Himmeli searches
for a minimum energy state that hopefully has as few edge crossing and node
overlaps as possible.

Positioning the vertices by simulation makes it difficult to determine when
the graph has settled to an optimal configuration. The limiting factor in this
process is time, since the rate at which large graphs converge to an
acceptable layout is slow and the simulation itself is computationally heavy.
For this reason, you can instruct Himmeli to stop after a fixed time quota by
invoking the \texttt{TimeLimit} parameter.

Suppose you have a very large graph and are uncertain about the time it takes
to produce an acceptable vertex layout. Most likely, you will not even know
what that layout would look like. It is therefore important that you can
continue a simulation if you think that the time quota was too short. In
effect, this means that you want to pass the coordinates of the previous
simulation run to the next so that Himmeli does not have to start from
scratch on each occasion.

\SubSection{Auxiliary input files}
By this time you have probably noticed that, in addition to the PS-file,
Himmeli creates three text files that can be used to quickly reconstruct
the layout. The accepted edges are stored to a new edge file, named according
to \texttt{GraphName}. The file contains edge colors, widths and labels, as
they appear in the figures. Similarly, the new vertex file has columns for the
visual variables, but it also contains the node coordinates, degrees and
strengths. The third text file is a new configuration file that instructs
Himmeli to use the auxiliary edge and vertex files to import a node layout
and visualize it accordingly.

To ensure that the graph topology remains unaffected, the
updated version of the configuration file conserves some of the original
instructions, but overrides those pertaining to visual cues and simulation
time. By setting a positive \texttt{TimeLimit}, you can start a new
simulation run that starts from the stored node layout. You can also use the
auxiliary files as a template for customization, note particularly the way
each variable can be fetched from a separate file if necessary.

Sometimes it is necessary to alter the visual outlook but to keep the
positions. For instance, if you wish to display the evolution of connections
as a series of figures, it is essential that the vertices do not move between
consecutive frames. Based on the previous discussion, this can be achieved by
first exporting the node coordinates from a network that represents the entire
series and then using this layout in every frame. By keeping the time quota at
zero, Himmeli will not update the initial positions and therefore only the
visual outlook of the edges and vertices is changed. Note, however, that if
some of the coordinates are missing, Himmeli will use arbitrary values in
their place.

\Section{Normalization and thresholding}
Usually, the weights of a graph are not constrained. The node-positioning
algorithm, however, works best if the edge weights are not much greater than
one and not very close to zero (non-positive weights are not accepted).
To tackle the problem, Himmeli has automatic normalization procedures that
map the edge weights to the desired value range. There are altogether four
options: you can set \texttt{EdgeWeightTransform} to fully automatic, linear
or logarithmic scaling, or rank-based transformation. 

\begin{figure}[h]
\begin{pspicture}(0, 0)(\textwidth, 9)
  \rput[lb](2.0, -1){\scalebox{0.8}{\epsfbox{figure7.eps}}}
  %\psframe(0, 0)(\textwidth, 9)
\end{pspicture}
\caption{The component in \FigRef{fig1} after thresholding by the median
  weight of the random graph. Only those edges with a weight above the median
  are drawn.}
\label{fig7}
\end{figure}

\begin{figure}[h]
\begin{pspicture}(0, 0)(\textwidth, 9)
  \rput[lb](2.0, -1){\scalebox{0.8}{\epsfbox{figure6.eps}}}
  %\psframe(0, 0)(\textwidth, 9)
\end{pspicture}
\caption{The component in \FigRef{fig1} after thresholding by the median
  weight of the random graph. Only those edges with a weight below the median
  are drawn.}
\label{fig6}
\end{figure}

Another typical need for a researcher is the thresholding of edge weights,
that is, you either want to exclude or not draw some of the edges according
to a constraint. For the former, you can use \texttt{EdgeWeightFilter} and to
remove edges from the visualization step only -- they are used when
positioning the nodes -- try the \texttt{EdgeWeightMask} instruction. Both
features have options that let you threshold by weight quantile or absolute
value (\FigRef{fig6} and \FigRef{fig7}).

\Section{Configuration interface}
\SubSection{File instructions}
\begin{description}
\item[\textnormal{\texttt{EdgeFile}}] \quad (required) \\
  Edge data file.
\item[\textnormal{\texttt{GraphName}}] \quad \\
  Base name for output files. The template is \\
  \textit{<base>\_<index>.eps} for graph components
  and \textit{<base>.<type>.txt} for the reconstruction files.
\item[\textnormal{\texttt{VertexFile}}] \quad \\
  Vertex data file. You do not have to supply data for every vertex. Only
  those vertices that are also present in the edge file will be recognized.
\end{description}

\SubSection{Structural instructions}
\begin{description}
\item[\textnormal{\texttt{EdgeHeadVariable}}] \quad (required) \\
  The column of head vertices in the edge or other file. 
\item[\textnormal{\texttt{EdgeLabelVariable}}] \quad \\
  The column of custom edge labels.
\item[\textnormal{\texttt{EdgeTailVariable}}] \quad (required) \\
  The column of tail vertices in the edge or other file. 
\item[\textnormal{\texttt{EdgeWeightFilter}}] \quad \\
  To threshold edges by weight before node positioning, three values must
  be supplied: the type of filter (\texttt{abs/frac}) and two limits.
  For example, to include only the largest 10\% edge fraction in the edge
  file, type \texttt{EdgeWeightFilter  frac  0.9  1.0}.
\item[\textnormal{\texttt{EdgeWeightMask}}] \quad \\
  Same as \texttt{EdgeWeightFilter} except that edges are not excluded until
  after node positioning.
\item[\textnormal{\texttt{EdgeWeightTransform}}] \quad \\
  Transform edge weights to optimal range for node positioning. There are
  three types of transformations: linear scaling (\texttt{lin}), logarithmic
  scaling (\texttt{log}) and rank-based normalization (\texttt{rank}).
  In addition, \texttt{auto} is a combination of the three and \texttt{off}
  uses raw weights. The default is \texttt{auto}. 
\item[\textnormal{\texttt{EdgeWeightVariable}}] \quad \\
  The column of numerical edge weights. Only positive values are accepted.
  If a weight variable is not supplied, each edge is given a uniform weight.
\item[\textnormal{\texttt{VertexNameVariable}}] \quad \\
  The column in the vertex  or other file that contains the list of vertex
  names.
\item[\textnormal{\texttt{VertexXVariable}}] \quad \\
  The column of horizontal positions in the vertex  or other file. The scale
  is normalized so that vertices should be at least a unit distance apart.
\item[\textnormal{\texttt{VertexYVariable}}] \quad \\
  The column of vertical positions in the vertex  or other file. The scale is
  normalized so that vertices should be at least a unit distance apart.
%\item[\textnormal{\texttt{VertexZVariable}}] \quad \\
%  The column of depth coordinates in the vertex file. The scale is normalized
%  so that vertices should be at least a unit distance apart.
\end{description}

\SubSection{Visualization instructions}
\begin{description}
\item[\textnormal{\texttt{EdgeColorVariable}}] \quad \\
  The column of color definitions in the edge file. Colors are determined by
  integer codes: 000000 for black, 505050 for gray, 999999 for white,
  990000 for red, 009900 for green, 000099 for blue, or any
  other between 000000 and 999999. 
\item[\textnormal{\texttt{EdgeColorInfo}}] \quad (multiple) \\
  Labels of specific edge colors to be depicted in the legend. Two value
  fields: label and code. Valid codes within $[000000, 999999]$.
\item[\textnormal{\texttt{EdgeWidthVariable}}] \quad \\
  The column of edge width multipliers. Non-negative values greater than 2 or
  smaller than 0.2 are truncated to the limits. A negative value will make
  the edge invisible. 
\item[\textnormal{\texttt{VertexColorInfo}}] \quad (multiple) \\
  Labels for specific vertex colors. See \texttt{EdgeColorInfo}. 
\item[\textnormal{\texttt{VertexColorVariable}}] \quad \\
  The color definitions in the vertex file, see \texttt{EdgeColorVariable}
  for usage. 
\item[\textnormal{\texttt{VertexLabelVariable}}] \quad \\
  The column that contains new name labels in the vertex file. The new names
  will be used only in the figure, they will not be used to identify
  vertices. 
\item[\textnormal{\texttt{VertexPatternInfo}}] \quad (multiple) \\
  Labels for specific patterns. Two value fields: code and label. Valid
  codes within $[1, 99]$, see \FigRef{fig5}.
\item[\textnormal{\texttt{VertexPatternVariable}}] \quad \\
  Individual patterns. Valid codes within $[1, 99]$, see \FigRef{fig5}.
\item[\textnormal{\texttt{VertexShapeInfo}}] \quad (multiple) \\
  Labels for specific vertex shapes. Usage is similar to
  \texttt{VertexPatternInfo}.
\item[\textnormal{\texttt{VertexShapeVariable}}] \quad \\
  The column of shape definitions in the vertex file. Shapes are determined
  by integer codes, see \FigRef{fig5} for details.
\item[\textnormal{\texttt{VertexSizeVariable}}] \quad \\
  Size multiplier for vertices.
\end{description}

\SubSection{Formatting and functional instructions}
\begin{description}
%\item[\textnormal{\texttt{3DMode}}] \quad \\
%  If set to \texttt{on}, the graph will be drawn as a stereo image. 
%  The default is \texttt{off}.
\item[\textnormal{\texttt{ArrowMode}}] \quad \\
  If set to \texttt{on}, each edge will be augmented with an arrow head.
  The default is \texttt{off}. 
\item[\textnormal{\texttt{BackgroundColor}}] \quad \\
  Canvas background color, see \texttt{EdgeColorVariable} on representing
  the three rgb components. The default is 999999 (white).
\item[\textnormal{\texttt{DecorationMode}}] \quad \\
  If set to \texttt{on}, Himmeli will decorate the graph based on edge weights
  and vertex strengths. The color palette cycles from yellow (low weight or
  strength) through green and blue to bright red (high weight or strength).
  The default is \texttt{off}.
\item[\textnormal{\texttt{Delimiter}}] \quad \\
  The character that separates the columns in the edge and vertex files. Use
  \texttt{tab} for the horizontal tabulator and \texttt{ws} for white space.
  The default is \texttt{tab}. 
\item[\textnormal{\texttt{DistanceUnit}}] \quad \\
  Set the standard distance between nodes on the canvas. This is not a zoom;
  it scales the coordinate system that is used when drawing the figures in
  such a way that the node symbols are not modified, only their mutual
  distances are changed. The default is 1.
\item[\textnormal{\texttt{FigureLimit}}] \quad \\
  The maximum allowed number of EPS image output files. Cannot be more than
  9999, default is 0.
\item[\textnormal{\texttt{FontSize}}] \quad \\
  Set the font size for vertex labels. This has no effect on the legend.
  The default is 10pt.
\item[\textnormal{\texttt{ForegroundColor}}] \quad \\
  Line color for contours and text, see \texttt{EdgeColorVariable} on
  representing the three rgb components. The default is 000000 (black).
\item[\textnormal{\texttt{IncrementMode}}] \quad \\
  If set to \texttt{on}, Himmeli will start simulating from the layout that
  is supplied in the vertex file. The default is \texttt{off}. 
\item[\textnormal{\texttt{LabelMode}}] \quad \\
  If set to \texttt{vertex}, the drawing is augmented with vertex names or
  user supplied labels. If set to \texttt{edge}, edge weights (or labels) are
  written on edge center points. If set to \texttt{on}, both previous
  options apply. The default is \texttt{vertex}.
\item[\textnormal{\texttt{LegendMode}}] \quad \\
  If set to \texttt{on} and there is visualization information available,
  a legend is drawn on each page on the main output file. The default is
  \texttt{off}. 
\item[\textnormal{\texttt{PageOrientation}}] \quad \\
  Figure orientation, can be \texttt{landscape} (default) or
  \texttt{portrait}.
\item[\textnormal{\texttt{PageSize}}] \quad \\
  Keyword for page sizes for the output documents. Possible choices are
  'letter', and 'a0' to 'a5' for the first value (main document). The second
  value refers to the component outputs (.eps files) with an additional option
  of automatic size 'auto'. The defaults are 'letter' and 'auto' for the
  first and second value, respectively.
\item[\textnormal{\texttt{TimeLimit}}] \quad \\
  The time quota in seconds for computing the node positions. If set to
  zero, no simulation is performed. The default is 10s. 
\item[\textnormal{\texttt{TreeMode}}] \quad \\
  If set to \texttt{on}, Himmeli will use only the maximum spanning tree
  instead of the full graph. The default is \texttt{off}.
\item[\textnormal{\texttt{VerboseMode}}] \quad \\
  If set to \texttt{on}, Himmeli will print all runtime messages. The
  default is \texttt{on}.
\end{description}
\end{document}
